\chapter{Opis za��czonej p�yty CD/DVD}
\label{chap:b}
Na za��czonej p�ycie CD/DVD zapisano:
\begin{enumerate}
  \item Plik o nazwie \texttt{W04N\_248869\_2021\_praca in�ynierska.pdf} stanowi�cy dokument pracy.
  \item Katalog o nazwie \texttt{biker} z plikami projektu aplikacji.
\end{enumerate}

W~celu konfiguracji �rodowiska uruchomieniowego, nale�y post�powa� wed�ug instrukcji zawartej w dodatku~\ref{chap:a}. Struktur� katalogu przedstawiono w tabeli~\ref{tab:filesstructure}. W trakcie pracy nad projektem, w katalogu g��wnym generowane s� dwa dodatkowe~katalogi (oraz ewentualne pliki tymczasowe):
\begin{enumerate}
  \item \texttt{.node\_modules} --- po zainstalowaniu bibliotek poleceniem \texttt{npm install}. Zawiera wykorzystywane biblioteki.
  \item  \texttt{build} --- po zbudowaniu aplikacji poleceniem \texttt{npm run build}. Zawiera kod zminifikowany i zoptymalizowany kod zbudowanej aplikacji. S�u�y do publikacji aplikacji w~us�udze hostingu.
\end{enumerate}
Oba te foldery ignorowane s� przez \texttt{.gitignore}.

W katalogu \texttt{components} opr�cz plik�w \texttt{App.tsx} i \texttt{App.less} znajduj� si� katalogi kt�rych nazwy odpowiadaj� nazwom poszczeg�lnych komponent�w. Nazwy tych katalog�w oraz pliki styli komponent�w zapisano notacj� \texttt{camelCase}, natomiast nazwy plik�w samych komponent�w --- notacj� \texttt{PascalCase} (zgodnie ze standardem \texttt{Reacta}).

\newpage
\begin{table}[htb] \small
  \centering
  \caption{Struktura plik�w projektu}
  \label{tab:filesstructure}
  \begin{tabularx}{\linewidth}{|l|l|}\hline
    Katalog g��wny & Katalog \texttt{src} (g��wny kod aplikacji)                   \\ \hline\hline
    \begin{minipage}{0.4\linewidth}
      \dirtree{%
        .1 {biker/}.
        .2 {.firebase/}.
        .3 {hosting.\emph{ID}.cache}.
        .2 {public/}.
        .3 {favicon.ico}.
        .3 {favicon.svg}.
        .3 {index.html}.
        .3 {logo.svg}.
        .3 {logo\_maskable.svg}.
        .3 {manifest.json}.
        .3 {robots.txt}.
        .2 {src/}.
        .3 {...}.
        .2 {.firebaserc}.
        .2 {.gitattributes}.
        .2 {.gitignore}.
        .2 {README.md}.
        .2 {craco.config.js}.
        .2 {firebase.json}.
        .2 {package-lock.json}.
        .2 {package.json}.
        .2 {package.json}.
        .2 {tsconfig.json}.
      }
    \end{minipage}
                   &
    \begin{minipage}{0.55\linewidth-2.5pt}\dirtree{%
        .1 {../}.
        .2 {src/}.
        .3 {assets/}.
        .4 {fonts/}.
        .5 {Road\_Rage.otf}.
        .4 {images/}.
        .5 {bg.jpg}.
        .4 {brand.svg}.
        .3 {components/}.
        .4 {\emph{componentName}/ (37 katalog�w)}.
        .5 {\emph{componentName}.less}.
        .5 {\emph{ComponentName}.tsx}.
        .4 {App.less}.
        .4 {App.tsx}.
        .3 {database/}.
        .4 {schema.tsx}.
        .3 {firebase/}.
        .4 {contexts/}.
        .5 {AuthContext.tsx}.
        .4 {hooks/}.
        .5 {useAuth.tsx}.
        .5 {useDatabase.tsx}.
        .4 {firebase.tsx}.
        .4 {firebaseUI.tsx}.
        .3 {global/}.
        .4 {geolocationMath.ts}.
        .4 {gpxBuilder.ts}.
        .4 {pages.ts}.
        .4 {statisticsFromatters.ts}.
        .4 {geolocationMath.ts}.
        .4 {geolocationMath.ts}.
        .3 {hooks/}.
        .4 {useActivityStatistics.tsx}.
        .4 {useGeolocation.tsx}.
        .4 {useInterval.tsx}.
        .4 {useNoSleep.tsx}.
        .4 {useTimeout.tsx}.
        .4 {useTracker.tsx}.
        .3 {index.tsx}.
        .3 {react-app-env.d.ts}.
        .3 {reportWebVitals.ts}.
        .3 {service-worker.ts}.
        .3 {serviceWorkerRegistration.ts}.
      }\end{minipage} \\ \hline
  \end{tabularx}
\end{table}

