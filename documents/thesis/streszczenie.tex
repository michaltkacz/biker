\pdfbookmark[0]{Streszczenie}{streszczenie.1}
%\phantomsection

\begin{abstract}
    Celem pracy jest stworzenie aplikacji internetowej udost�pniaj�ce fundamentalne funkcje komputera rowerowego. Aplikacja ma by� przystosowana do dzia�ania zar�wno na urz�dzeniach desktopowych jak i urz�dzeniach mobilnych. Do podstawowych funkcji aplikacji nale��: uwierzytelnianie u�ytkownika, nagrywanie �lad�w przejechanych tras rowerowych z wykorzystaniem modu�u GPS (ang.~\emph{Global Positioning System}), przegl�danie zapisanych �lad�w, analiza statystyk zapisanych �lad�w oraz mo�liwo�� importu i eksportu danych z wykorzystaniem standardu GPX (ang.~\emph{GPS Exchange Format}). Wyr�nikiem tej aplikacji na tle dost�pnych na rynku rozwi�za� jest to, �e jest to aplikacja internetowa typu PWA (ang.~\emph{Progressive Web App}), co umo�liwia zainstalowanie jej na urz�dzeniach mobilnych i sprawienie wra�enia aplikacji natywnej. Niniejszy dokument zawiera podstawowe wyja�nienie czym jest komputer rowerowy oraz przedstawia szczeg�y techniczne projektu.
\end{abstract}
% \mykeywords{aplikacja internetowa, komputer rowerowy, nawigacja rowerowa, GPS, GPX}\\
% Dobrze by�oby skopiowa� s�owa kluczowe do metadanych dokumentu pdf (w pliku Dyplom.tex)
% Niestety, zaimplementowane makro nie robi tego z automatu, wi�c pozostaje kopiowanie r�czne.

{
\selectlanguage{english}
\begin{abstract}
    The goal of the work is to create a web application providing the fundamental functions of a cycling computer. The application should be available for both desktop and mobile devices. The basic functions of the application shall be as follows: user authentication, cycling tracks recording using the GPS (\emph{Global Positioning System}), recorded tracks browsing, recorded tracks statistics analysis and the data import and export in the GPX (\emph{GPS Exchange Format}) standard format. The main feature that distinguishes this application from already available products is the use of PWA (\emph{Progressive Web App}), which makes it possible to install web application on mobile devices and simulate the experience of a native mobile application. This document provides a basic explanation of what cycling computer is and provides technical details of the project.
\end{abstract}
% \mykeywords{web application, cycling computer, cycling navigation, GPS, GPX}
}
