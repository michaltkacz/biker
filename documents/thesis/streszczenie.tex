\pdfbookmark[0]{Streszczenie}{streszczenie.1}
%\phantomsection

\begin{abstract}
    Celem pracy jest stworzenie aplikacji internetowej udost�pniaj�ce fundamentalne funkcje komputera rowerowego. Aplikacja ma by� przystosowana zar�wno na urz�dzenia desktopowe jak i urz�dzenia mobilne. Do podstawowych funkcji aplikacji nale��: autoryzacja u�ytkownika, nagrywanie �lad�w przejechanych tras rowerowych z wykorzystaniem modu�u GPS, przegl�danie zapisanych �lad�w, analiza statystyk zapisanych �lad�w oraz mo�liwo�� importu i eksportu danych z wykorzystaniem standardu GPX. Wyr�nikiem tej aplikacji na tle dost�pnych na rynku rozwi�za� jest to, �e jest to aplikacja internetowa typu PWA, co umo�liwia zainstalowanie jej na urz�dzeniach mobilnych i sprawienie wra�enia aplikacji natywnej. Niniejszy dokument zawiera podstawowe wyja�nienie czym jest komputer rowerowy oraz przedstawia szczeg�y techniczne projektu.
\end{abstract}
\mykeywords{aplikacja internetowa, komputer rowerowy, nawigacja rowerowa, rower, GPS, GPX}\\
% Dobrze by�oby skopiowa� s�owa kluczowe do metadanych dokumentu pdf (w pliku Dyplom.tex)
% Niestety, zaimplementowane makro nie robi tego z automatu, wi�c pozostaje kopiowanie r�czne.

{
\selectlanguage{english}
\begin{abstract}
    content
\end{abstract}
\mykeywords{}
}
